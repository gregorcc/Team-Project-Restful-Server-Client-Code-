\documentclass[12pt]{article}
\usepackage[english]{babel}
\usepackage{listings}
\usepackage[dvipsnames]{xcolor}


%this code came from https://github.com/cansik/kotlin-latex-listing
\lstdefinelanguage{Kotlin}{
  keywords={package, as, as?, typealias, this, super, val, var, fun, for, null, true, false, is, in, throw, return, break, continue, if, try, else, while, do, when, class, interface, enum, object, companion, override, public, private, open, sealed, get, set, import, abstract, vararg, expect, extend, actual, where, suspend, data, internal, dynamic, final, by, find, close},
  keywordstyle=\color{Fuchsia}\bfseries,
  ndkeywords={@Deprecated, @JvmName, @JvmStatic, @JvmOverloads, @JvmField, @JvmSynthetic, Iterable, Int, Long, Integer, Short, Byte, Float, Double, String, Runnable, Array, T, Unit, Observable,View, Controller, Fragment},
  ndkeywordstyle=\color{Red}\bfseries,
  emph={print, println, return@, forEach, map, mapNotNull, first, filter, firstOrNull, lazy, delegate, init, with, apply, sumBy, let, max, min, sum, maxBy, minBy, joinToString, toString, action},
  emphstyle={\color{BurntOrange}},
  emph={[2]it, key, value, self, mapKey, mapValue, counts, main, root, vbox, hbox, button, label, Color, Font},
  emphstyle={[2]\color{Aquamarine}\bfseries},
  identifierstyle=\color{black},
  sensitive=true,
  commentstyle=\color{SpringGreen}\ttfamily,
  comment=[l]{//},
  morecomment=[s]{/*}{*/},
  stringstyle=\color{OliveGreen}\ttfamily,
  morestring=[b]",
  morestring=[s]{"""*}{*"""},
  columns=flexible,
  tabsize=4,
  breaklines=true,
  backgroundcolor=\color{gray!6},
}

\lstset{
basicstyle=\ttfamily,
columns=flexible,
tabsize=4,
breaklines=true,
backgroundcolor=\color{gray!10}
}

\begin{document}
\section{TornadoFX}

\subsection{Basic Structure}
\begin{lstlisting}
Components:
	-> Application: the entry point is "App" and specifies the initial "View"
		-> Stage: 
		-> Scene:
			-> Model: The business code layer that holds core logic and data
			-> View: The visual display with various input and output controls
			-> Cotroller:The "middleman" mediating events between the Model and the View
			-> Fragment: The pop-ups
\end{lstlisting}
\subsection{View}
Many components are automatically maintained as singletons.

\begin{lstlisting}[language=Kotlin]
Extend App: class MyApp: App(MyView::class)
\end{lstlisting}
View contains display logic and layout of nodes, it's singleton.
\\
When a View is declared, there must be a root property which can be any Parent type, and that will hold the View's content.
\\
Use the "plus assign" += operators to add children, such as a Button and Label.

\begin{lstlisting}[language=Kotlin]
class MyView: View() {
	override val root = VBox()
	init {
		root += Button("Press Me")
		root += Label("Waiting")
	}
}
\end{lstlisting}
	A BoderPane contains function TopView and BottomView, which can be embed by inject() delegate property to load other View. Then Each "child" View's root to assign them to the BorderPane.

\begin{lstlisting}[language=Kotlin]
import javafx.scene.control.Label
import javafx.scene.layout.BorderPane
import tornadofx.*
class MasterView: View() {
	val topView: TopView by inject()
	val bottomView: BottomView by inject()
	//Here is a alter way: use find to insert a View
	val topView = find(TopView::class)
	val bottomView = find(BottomView::class)	
	
	override val root = BorderPane()
	init {
		with(root) {
			top = topView.root
			bottom = bottomView.root
			}
		}
	}
class TopView: View() {
	override val root = Label("Top View")
}
class BottomView: View() {
	override val root = Label("Bottom View")
}

\end{lstlisting}
Here is another alter way to use view builder to insert.

\begin{lstlisting}[language=Kotlin]
import javafx.scene.control.Label
import tornadofx.*
class MasterView : View() {
	override val root = borderpane {
		top(TopView::class)
		bottom(BottomView::class)
	}
}
class TopView: View() {
	override val root = Label("Top View")
}
class BottomView: View() {
	override val root = Label("Bottom View")
}
\end{lstlisting}

\subsection{Controllers}
Controllers will help to finish a specific task in the back.
\begin{lstlisting}[language=Kotlin]
import tornadofx.*
class MyView : View() {
	val controller: MyController by inject()
	override val root = vbox {
		label("Input")
		val inputField = textfield()
		button("Commit") {
			action {
				controller.writeToDb(inputField.text)
				inputField.clear()
			}
		}
	}
}

// this controller helps to write data
class MyController: Controller() {
	fun writeToDb(inputValue: String) {
	println("Writing $inputValue to database!")
	}
}
\end{lstlisting}


\begin{lstlisting}[language=Kotlin]
import javafx.collections.FXCollections
import tornadofx.*
class MyView : View() {
	val controller: MyController by inject()
	override val root = vbox {
		label("My items")
		listview(controller.values)
	}
}
// this controller helps to write data into local dataset.
class MyController: Controller() {
	val values = FXCollections.observableArrayList("Alpha","Beta","Gamma","Delta")
}
\end{lstlisting}

\begin{lstlisting}[language=Kotlin]
val textfield = textfield()
button("Update text") {
	action {
		runAsync {
			myController.loadText()
		} ui { loadedText ->
			textfield.text = loadedText
		}
	}
}
\end{lstlisting}
Here is a fragment, which is a a way to do pop-up.
\begin{lstlisting}[language=Kotlin]
import javafx.stage.StageStyle
import tornadofx.*
class MyView : View() {
	override val root = vbox {
		button("Press Me") {
			action {
				find(MyFragment::class).openModal(stageStyle = StageStyle.UTILITY)
				}
			}
		}
	}
class MyFragment: Fragment() {
	override val root = label("This is a popup")
}
\end{lstlisting}
This piece of code will open a new window.
\begin{lstlisting}[language=Kotlin]
button("Open editor") {
	action {
		openInternalWindow(Editor::class)
	}
}
\end{lstlisting}


\begin{lstlisting}[language=Kotlin]
open(), close() \\ Gives the access to open or close window.
findParentOfType(InternalWindow::class)
\end{lstlisting}
By using replaceWith, we can change windows.
\begin{lstlisting}[language=Kotlin]
button("Go to MyView2") {
	action {
		replaceWith(MyView2::class)
	}
}
\end{lstlisting}
We can even create some animation.
\begin{lstlisting}[language=Kotlin]
replaceWith(MyView1::class, ViewTransition.Slide(0.3.seconds, Direction.LEFT))
\end{lstlisting}
onDuck() and onUnDuck() will set action will View being replaced.\\\\
We can also passing parameters to Views, and we need specify configuring parameters for the target View.
\begin{lstlisting}[language=Kotlin]
fun editCustomer(customer: Customer) {
	find<CustomerEditor>(mapOf(CustomerEditor::customer to customer)).openWindow()
}
\end{lstlisting}
Perform null check for the right parameters.
\begin{lstlisting}[language=Kotlin]
class CustomerEditor : Fragment() {
	init {
		val customer = params["customer"] as? Customer
		if (customer != null) {
		...
		}
	}
}
\end{lstlisting}
View has a property called $primaryStage$ that allows you to manipulate properties of the Stage backing it, such as window size. Any View or Fragment that were opened via openModal will also have a $modalStage$ property available.

\subsection{Accessing Resources}
Lots of JavaFX APIs takes resources as a URL or the toExternalForm of an URL. To retrieve a resource url one would typically write something like:

\begin{lstlisting}[language=Kotlin]
val myAudioClip = AudioClip(MyView::class.java.getResource("mysound.wav").toExternalForm())
\end{lstlisting}
However, in TornadoFX every Component has resources function!
\begin{lstlisting}[language=Kotlin]
val myAudiClip = AudioClip(resources["mysound.wav"])

val myResourceURL = resources.url("mysound.wav")

val myJsonObject = resources.json("myobject.json")

val myJsonArray = resources.jsonArray("myarray.json")

val myStream = resources.stream("somefile")
\end{lstlisting}
Keyword with can be use to help assign components.
\begin{lstlisting}[language=Kotlin]
import javafx.scene.control.Button
import javafx.scene.layout.VBox
import tornadofx.*
class MyView : View() {
    override val root = VBox()
    init {
        with(root) {
            this += Button("Press Me")
        }
        // OR
        root.apply {
			this += Button("Press Me").apply {
                textFill = Color.RED
                action { println("Button pressed!") }
            }
		}
    }
}
\end{lstlisting}
The VBox (or any targetable component) has an extension
function called button(). It accepts a text argument and an optional closure targeting a Button it will instantiate.\\\\
When this function is called, it will create a Button with the specified text, apply the closure to it, add it to the VBox it was called on, and then return it.
\begin{lstlisting}[language=Kotlin]
import tornadofx.*
import javafx.scene.control.TextField

class tView : View() {
    var firstNameField: TextField by singleAssign()
    var lastNameField: TextField by singleAssign() 
    /*recommended you use the singleAssign() delegates to
 ensure the properties are only assigned once.*/
    override val root = vbox {
        hbox {
            label("First Name")
            firstNameField = textfield()
        }
        hbox {
            label("Last Name")
            lastNameField = textfield()
        }
        button("LOGIN") {
            useMaxWidth = true
            action {
                println("Logging in as ${firstNameField.text} ${lastNameField.text}")
            }
        }
    }
}
\end{lstlisting}

\section{Builders for Basic Controls}
\subsection{Button}
Button can optionally pass a text argument and a Button.() -> Unit lambda to modify its properties.
\subsection{Label}
label() extension function can be called to add a Label to a given Pane. Optionally you can provide a text (of type String or Property<String> ), a graphic (of type Node or ObjectProperty<Node> ) and a Label.() -> Unit lambda to modify its properties 
\begin{lstlisting}[language=Kotlin]
label("Lorem ipsum") {
	textFill = Color.BLUE
}
\end{lstlisting}
\subsection{TextField}
\begin{lstlisting}[language=Kotlin]
textfield("Input something") {
	textProperty().addListener { obs, old, new ->
		println("You typed: " + new)
	}
}
\end{lstlisting}
\subsection{Password Mode}
\begin{lstlisting}[language=Kotlin]
passwordfield("password123") {
	requestFocus()
}
\end{lstlisting}
Check Box
\begin{lstlisting}[language=Kotlin]
val booleanProperty = SimpleBooleanProperty()

checkbox("Admin Mode", booleanProperty).action {
	println(isSelected)
}
\end{lstlisting}
ComboBox

\begin{lstlisting}[language=Kotlin]
val texasCities = FXCollections.observableArrayList("Austin",
"Dallas","Midland", "San Antonio","Fort Worth")

combobox<String> {
	items = texasCities
}
\end{lstlisting}

\begin{lstlisting}[language=Kotlin]
val texasCities = FXCollections.observableArrayList("Austin",
"Dallas","Midland","San Antonio","Fort Worth")

val selectedCity = SimpleStringProperty()

combobox(selectedCity, texasCities)
\end{lstlisting}
\subsection{ToggleButton}
\begin{lstlisting}[language=Kotlin]
togglebutton("OFF") {
	action {
		text = if (isSelected) "ON" else "OFF"
	}
}
\end{lstlisting}

\begin{lstlisting}[language=Kotlin]
togglebutton {
    val stateText = selectedProperty().stringBinding {
    	if (it == true) "ON" else "OFF"
    }
    textProperty().bind(stateText)
}
\end{lstlisting}

\subsection{Hyperlink and text}
HyperLink

\begin{lstlisting}[language=Kotlin]
hyperlink("Open File").action { println("Opening file...") }
\end{lstlisting}
Text

\begin{lstlisting}[language=Kotlin]
text("Veni\nVidi\nVici") {
	fill = Color.PURPLE
	font = Font(20.0)
}
\end{lstlisting}
Textflow

\begin{lstlisting}[language=Kotlin]
textflow {
	text("Tornado") {
		fill = Color.PURPLE
		font = Font(20.0)
	}
	text("FX") {
		fill = Color.ORANGE
		font = Font(28.0)
	}
}
\end{lstlisting}
Tooltip

\begin{lstlisting}[language=Kotlin]
button("Commit") {
	tooltip("Writes input to the database") {
		font = Font.font("Verdana")
	}
}
\end{lstlisting}
Shortcut

\begin{lstlisting}[language=Kotlin]
shortcut("Ctrl+Y")) {
	doSomething()
}
\end{lstlisting}

\begin{lstlisting}[language=Kotlin]
button("Save") {
	action { doSave() }
	shortcut("Ctrl+S")
}
\end{lstlisting}
ProgressBar
\begin{lstlisting}[language=Kotlin]
progressbar(completion) {
    progressProperty().addListener {
        obsVal, old, new -> print("VALUE: $new")
    }
}
\end{lstlisting}




\section{Data Controls}

ListView is similar to combobox, but it can scroll, with option to select multiple line.
\begin{lstlisting}[language=Kotlin]
listview<String> {
	items.add("Alpha")
	items.add("Beta")
	items.add("Gamma")
	items.add("Delta")
	items.add("Epsilon")
	selectionModel.selectionMode = SelectionMode.MULTIPLE
}
\end{lstlisting}

\begin{lstlisting}[language=Kotlin]
val greekLetters = listOf("Alpha","Beta",
"Gamma","Delta","Epsilon").observable()

listview(greekLetters) {
	selectionModel.selectionMode = SelectionMode.MULTIPLE
}
\end{lstlisting}
TableView

\begin{lstlisting}[language=Kotlin]
class Person(val id: Int, val name: String, val birthday:LocalDate) {
	val age: Int get() = Period.between(birthday,LocalDate.now()).years
}
\end{lstlisting}

\begin{lstlisting}[language=Kotlin]
private val persons = listOf(
	Person(1,"Samantha Stuart",LocalDate.of(1981,12,4)),
	Person(2,"Tom Marks",LocalDate.of(2001,1,23)),
	Person(3,"Stuart Gills",LocalDate.of(1989,5,23)),
	Person(3,"Nicole Williams",LocalDate.of(1998,8,11))
).observable()
\end{lstlisting}

\begin{lstlisting}[language=Kotlin]
tableview(persons) {
	column("ID",Person::id)
	column("Name", Person::name)
	column("Birthday", Person::birthday)
	column("Age",Person::age)
}
//readonlyColumn() is immutable.
\end{lstlisting}

\begin{lstlisting}[language=Kotlin]
class Person(id: Int, name: String, birthday: LocalDate){
	var id by property(id)
	fun idProperty() = getProperty(Person::id)
	var name by property(name)
	fun nameProperty() = getProperty(Person::name)
	var birthday by property(birthday)
	fun birthdayProperty() = getProperty(Person::birthday)
	val age: Int get() = Period.between(birthday, LocalDate.now()).years
}
//Delegation
override val root = tableview(persons) {
isEditable = true
	column("ID",Person::idProperty).makeEditable()
	column("Name", Person::nameProperty).makeEditable()
	column("Birthday", Person::birthdayProperty).makeEditable()
	column("Age",Person::age)
}
\end{lstlisting}
Extension Function: useTextField(), useComboBox(), useChoiceBox(), useCheckBox(), useProgressBar() 
\\\\
Make the property private 
\begin{lstlisting}[language=Kotlin]
private val nameProperty = SimpleStringProperty(name)
fun nameProperty() = nameProperty
var name by nameProperty
\end{lstlisting}
cellFormat()

\begin{lstlisting}[language=Kotlin]
tableview(persons) {
    column("ID", Person::id)
    column("Name", Person::name)
    column("Birthday", Person::birthday)
    column("Age", Person::age).cellFormat {
        text = it.toString()
        style {
            if (it < 18) {
                backgroundColor += c("#8b0000")
                textFill = Color.WHITE
            } else {
                backgroundColor += Color.WHITE
                textFill = Color.BLACK
            }
        }
    }
}
\end{lstlisting}
Define the Property as the parent property, so it will update after View refreshed.
\begin{lstlisting}[language=Kotlin]
column("Parent name", Person::parentProperty).cellFormat{
    textProperty().bind(it.parentProperty.value.nameProperty)
}
\end{lstlisting}
This will reflect change immediately.
\begin{lstlisting}[language=Kotlin]
column<Person, String>("Parent name", { it.value.parentProperty.select(Person::nameProperty) })
\end{lstlisting}
\subsection{Row Expanders}
\begin{lstlisting}[language=Kotlin]
class Region(val id: Int, val name: String, val country: String, val branches: Observa
             bleList<Branch>)
class Branch(val id: Int, val facilityCode: String, val city: String, val stateProvince
: String)
val regions = listOf(
        Region(1,"Pacific Northwest", "USA",listOf(
                Branch(1,"D","Seattle","WA"),
                Branch(2,"W","Portland","OR")
        ).observable()),
        Region(2,"Alberta", "Canada",listOf(
                Branch(3,"W","Calgary","AB")
        ).observable()),
        Region(3,"Midwest", "USA", listOf(
                Branch(4,"D","Chicago","IL"),
                Branch(5,"D","Frankfort","KY"),
                Branch(6, "W","Indianapolis", "IN")
        ).observable())
).observable()
\end{lstlisting}

\begin{lstlisting}[language=Kotlin]
override val root = tableview(regions) {
    column("ID",Region::id)
    column("Name", Region::name)
    column("Country", Region::country)
    rowExpander(expandOnDoubleClick = true) {
        paddingLeft = expanderColumn.width
        tableview(it.branches) {
            column("ID",Branch::id)
            column("Facility Code",Branch::facilityCode)
            column("City",Branch::city)
            column("State/Province",Branch::stateProvince)
        }
    }
}
\end{lstlisting}
Add property to rowExpander
\begin{lstlisting}[language=Kotlin]
val expander = rowExpander(true) { ... }
\end{lstlisting}

\subsection{TreeView}

\begin{lstlisting}[language=Kotlin]
data class Person(val name: String, val department: String)
val persons = listOf(
        Person("Mary Hanes","Marketing"),
        Person("Steve Folley","Customer Service"),
        Person("John Ramsy","IT Help Desk"),
        Person("Erlick Foyes","Customer Service"),
        Person("Erin James","Marketing"),
        Person("Jacob Mays","IT Help Desk"),
        Person("Larry Cable","Customer Service")
)
\end{lstlisting}

\begin{lstlisting}[language=Kotlin]
// Create Person objects for the departments
// with the department name as Person.name
val departments = persons
        .map { it.department }
        .distinct().map { Person(it, "") }
treeview<Person> {
// Create root item
    root = TreeItem(Person("Departments", ""))
// Make sure the text in each TreeItem is the name of the Person
    cellFormat { text = it.name }
// Generate items. Children of the root item will contain departments
    populate { parent ->
        if (parent == root) departments else persons.filter { it.department == parent.
                value.name }
    }
}
\end{lstlisting}

\end{document}